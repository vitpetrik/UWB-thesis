
%!TEX ROOT = ctutest.tex

\ProvidesExplFile{ctuth-pkg.tex}%
	{2016/06/15}{0.1 t1606152353}%MY TIMESTAMP HERE {0.1}
	{"Packages"\space part\space of\space the\space class\space ctuthesis}

% A lot of the code here is not latex3, but rather the good'ol latex2e code. The reason is that
% it heavily depends on l2e packages, and we decided not to mix the code together too much.

\RequirePackage{lmodern}
\RequirePackage[T1]{fontenc}
\RequirePackage{microtype}
\RequirePackage{graphicx}
\RequirePackage{pdfpages}




%%% BABEL -- LANGUAGE HANDLING

% The loading of the languages is a bit wicked, but it works this way. We load the main language once more
% to make it the default one.
\RequirePackage[\seq_use:Nn \g_ctuthesis_languages_seq {,},\g_ctuthesis_field_mainlanguage_tl]{babel}

% Used for setting title, main or second language
\NewDocumentCommand \selectctulanguage { m } {
	\ctuthesis_field_exists:nTF { #1 language }
		{ \exp_args:Nx \selectlanguage { \ctuthesis_field_use_default:nn { #1 language } { } } }
		{ \msg_error:nnx { ctuthesis } { unknown-field } { #1 language } }
}







%%% COLORS, TEXT FLOW

\RequirePackage{xcolor}

% Define the colors, first in monochrome, then in colour.
\ctuifswitch { savetoner } {
	% savetoner: true
	\definecolor { ctublue } { CMYK } { 100, 43, 0, 0 }
	\definecolor { ctulightblue } { RGB }{ 172, 214, 238 }
	\colorlet    { ctubluetext } { ctublue }
	\colorlet    { ctubluerule } { ctulightblue }
	\colorlet    { ctubluedarkbg } { white }
	\colorlet    { ctubluedarkfg } { ctublue!50!black }
	\colorlet    { ctubluedot } { white }%ctulightblue }
	\colorlet    { ctubluebg } { white }
	\colorlet    { ctulstbg } { ctubluebg }
	\colorlet    { ctublueitemi } { ctublue }
	\colorlet    { ctublueitemii } { ctulightblue }
	\colorlet    { ctugray } { gray }
	\definecolor { ctuorange } { Hsb } { 22, 0.5, 1 }
} {
	\ctuifswitch { monochrome } {
		% savetoner: false, monochrome: true
		\definecolor { ctublue } { gray } { 0.8 }
		\definecolor { ctulightblue } { gray } { 0.9 }
		\definecolor { ctubluetext } { gray } { 0.4 }
		\definecolor { ctubluerule } { gray } { 0.4 }
		\definecolor { ctubluedarkbg } { gray } { 0.2 }
		\definecolor { ctubluedarkfg } { gray } { 1 }
		\definecolor { ctubluedot } { gray } { 0.9 }
		\definecolor { ctubluebg } { gray } { 0.95 }
		\definecolor { ctulstbg } { gray } { 0.95 }
		\definecolor { ctublueitemi } { gray } { 0.4 }
		\definecolor { ctublueitemii } { gray } { 0.4 }
		\definecolor { ctugray } { gray } { 0.5 }
		\definecolor { ctuorange } { gray } { 0.9 }
	} {
		% savetoner: false, monochrome: false
		\definecolor { ctublue } { cmyk } { 1, .43, 0, 0 }
		\definecolor { ctulightblue } { cmyk }{ .3, .13, 0, 0 }
		\colorlet    { ctubluetext } { ctublue }
		\colorlet    { ctubluerule } { ctublue }
		\colorlet    { ctubluedarkbg } { ctublue }
		\colorlet    { ctubluedot } { ctulightblue }
		\colorlet    { ctubluebg } { ctulightblue!50!white }
		\colorlet    { ctulstbg } { ctulightblue!50!white }
		\colorlet    { ctublueitemi } { ctublue }
		\colorlet    { ctublueitemii } { ctulightblue }
		\colorlet    { ctugray } { gray }
		\definecolor { ctuorange } { cmyk } { 0, .51, 1, 0 }
	}
}

\setlength{\parskip}{0pt plus 1pt}

\setlength{\parindent}{11.2pt}






%%% FRONT MATTER --- TWOCOLUMN HANDLING

% This is a "ToC macro" for the twocolumn context
% Arguments: 1) suffix, 2) ToC name
\cs_new_protected:Nn \ctuthesis_maketoc_twocol:nn {
	% Start the chapter
	\chapter* { #2 }
	% locally make rugged pages
	\begingroup
	\raggedbottom
	% We use this trick to add nothing, but also to make sure that
	% any \addvspace for less than 20pt is going to be ignored.
	\vspace*{-20pt}\addvspace{20pt}
	% call the original \@starttoc
	\@starttoc { #1 }
	\newpage
	\endgroup
}

% Environment for the twocolumn frontmatter. It can appear more times in a row
% and should handle it nicely
\NewDocumentEnvironment{ twocolumnfrontmatterpage } { } {
	% Set the page geometry to the title one (it's a bit wider)
	\ctuthesis_geometry_title:
	\newpage
	\pagestyle{twocol}
	% Start the twocolumn.
	\twocolumn[]\relax
	% We are narrow, so we wanna be sloppy
	\sloppy
	% Chapter behaves differently. Both the [] argument and the star * are ignored.
%	\DeclareDocumentCommand\chapter{ s o m }{\relax
%		% Vertical adjustment
%		\vspace* { -\baselineskip }
%		\nointerlineskip
%		% Zero measure vertical box, the contents are raised by something and typeset
%		\vbox to 0pt{\noindent\leavevmode\smash{\raise6pt\hbox to \linewidth{%
%			% In the first column its right-aligned, in the second left-aligned.
%			% It can't be more than one line.
%			\if@firstcolumn\hfil\fi
%			\color{ctubluetext}\LARGE\bfseries\sffamily##3
%			\if@firstcolumn\else\hfil\fi
%		}}}
%		\everypar{\noindent}
%	}
	\DeclareDocumentCommand\chapter{ s o m }{\relax
		% Compute the vertical adjustment
		\skip_gset:Nn \g_ctuthesis_tempa_skip { - \box_ht:N \strutbox }
		{ \LARGE \skip_gsub:Nn \g_ctuthesis_tempa_skip { \box_ht:N \strutbox + \box_dp:N \strutbox } }
		% Space used if not at the beginning of a column
		\skip_vertical:n { 12pt - \g_ctuthesis_tempa_skip }
		% Add the space below the chapter title to the adjustment
		\skip_gsub:Nn \g_ctuthesis_tempa_skip { 2pt }
		% Vertical adjustment
		\vspace* { \g_ctuthesis_tempa_skip }
		% Zero measure vertical box, the contents are raised by something and typeset
		\noindent \leavevmode
		\hbox_to_wd:nn \linewidth {
			% In the first column its right-aligned, in the second left-aligned.
			% It can't be more than one line.
			\if@firstcolumn\hfil\fi
			\leavevmode\color{ctubluetext}\LARGE\strut\bfseries\sffamily\fontspec{Technika}##3
			\if@firstcolumn\else\hfil\fi
		}\par
		\skip_vertical:n { 2pt }
		\everypar{\noindent}
	}
	\DeclareDocumentCommand \tableofcontents {  }{ \ctuthesis_maketoc_twocol:nn { toc } { \contentsname } }
	\DeclareDocumentCommand \listoffigures {  }{ \ctuthesis_maketoc_twocol:nn { lof } { \listfigurename } }
	\DeclareDocumentCommand \listoftables {  }{ \ctuthesis_maketoc_twocol:nn { lot } { \listtablename } }
} {
	\ctuthesis_geometry_plain:
	\clearpage
	\onecolumn
	\pagestyle{plain}
}






%%% LANGUAGES

% For every of the languages below, if it's initialized, add the
% custom captions of the class
\@ifundefined{captionsenglish}{}{\g@addto@macro\captionsenglish{\ctucaptionsenglish}}
\@ifundefined{captionsczech}{}{\g@addto@macro\captionsczech{\ctucaptionsczech}}
\@ifundefined{captionsslovak}{}{\g@addto@macro\captionsslovak{\ctucaptionsslovak}}





%%% GEOMETRY

% The ratios of inner:outer margins
% Needs to be separate as this to make the colon ":" have the right catcode
\cs_new:Nx \ctuthesis_geometry_hmarginratio: {4\string :6}

% Target textwidth is 33*basesize (10pt to 12pt)
\dim_const:Nn \g_ctuthesis_geometry_textwidth_dim { \tl_use:N \g_ctuthesis_fontsize_tl * 33 }

% Text occupies 75% of the height of the page by default, less if layout-short is active
% The vmargin ratio is set here in the same manner as the hmargin one above
\ctuifswitch { layout-short } {
	\dim_const:Nn \g_ctuthesis_geometry_textheight_dim { 1.5\g_ctuthesis_geometry_textwidth_dim }
	\cs_new:Nx \ctuthesis_geometry_vmarginratio: {4\string :3}
} {
	\dim_const:Nn \g_ctuthesis_geometry_textheight_dim { 0.75\paperheight }
	\cs_new:Nx \ctuthesis_geometry_vmarginratio: {5\string :6}
}

% Loading the geometry package.
% Almost the same setting appears again in \ctuthesis_geometry_plain:
\RequirePackage [ a4paper ,
	hmarginratio = \ctuthesis_geometry_hmarginratio: ,
	textwidth = \g_ctuthesis_geometry_textwidth_dim ,
 	textheight = \g_ctuthesis_geometry_textheight_dim ,
	vmarginratio = \ctuthesis_geometry_vmarginratio: , ignoreheadfoot ,
	% Should be at least \baselineskip
	headheight = 16pt ,
	marginparsep = 10pt ,
] { geometry }

% Inner margin is 1in + \oddsidemargin. We wanna preserve this value through all
% the different page geometries
\dim_const:Nn \g_ctuthesis_geometry_innermargin_dim { 1in + \the\oddsidemargin }

% The frontmatter pages are a bit wider
\cs_new_protected:Nn \ctuthesis_geometry_title: {
	\newgeometry {
		% Use the stored innermargin value
		inner = \g_ctuthesis_geometry_innermargin_dim ,
		% Increase the width by 20%, but do not exceed 420pt
		textwidth = \dim_min:nn { 420pt } { 1.2 \g_ctuthesis_geometry_textwidth_dim } ,
	 	textheight = \g_ctuthesis_geometry_textheight_dim ,
		vmarginratio = \ctuthesis_geometry_vmarginratio: ,
		headheight = 16pt ,
		ignoreheadfoot ,
		columnsep=30pt ,
	}
}

\cs_new_protected:Nn \ctuthesis_geometry_plain: {
	\newgeometry {
		% Use the stored innermargin value
		inner = \dim_use:N \g_ctuthesis_geometry_innermargin_dim ,
		% Otherwise it's the same as in \RequirePackage[...]{geometry}
		textwidth = \dim_use:N \g_ctuthesis_geometry_textwidth_dim ,
	 	textheight = \g_ctuthesis_geometry_textheight_dim ,
		vmarginratio = \ctuthesis_geometry_vmarginratio: , ignoreheadfoot ,
		headheight = 16pt ,
		marginparsep = 10pt ,
	}
	% Compute the marginparwidth so that we leave 40pt from the page boundary
	\marginparwidth \dimexpr \evensidemargin + 1in - \marginparsep - 40pt
}

% Re-initialize the geometry
\ctuthesis_geometry_plain:






%%% TITLES

\usepackage{titlesec}
% two basic parameters: the rule width and distance
\newlength \ctu@title@rulewidth
\newlength \ctu@title@rulesep
\setlength \ctu@title@rulewidth {11pt}
\setlength \ctu@title@rulesep {11pt}
% box used all the time
\newbox\ctu@title@box
% macro that boxifies the header and adds the rule
\long\def\ctu@title@boxify#1#2#3#4{
	\fontspec{Technika}
	\bfseries
	\setbox\ctu@title@box\vbox{\hsize\dimexpr\linewidth-\ctu@title@rulewidth-\ctu@title@rulesep\relax
		\vskip#1
		\raggedright
		#3
		\vskip#2
	}
	#4
	\noindent\begin{tabular}{@{}l@{\hspace*{\ctu@title@rulesep}}l@{}}
		\color{ctubluerule}\rule[-\dp\ctu@title@box]{\ctu@title@rulewidth}{\dimexpr\ht\ctu@title@box+\dp\ctu@title@box}
	&
		\box\ctu@title@box
	\end{tabular}\hspace*{-10pt}
	\par
}
\long\def\ctu@title@boxify@x#1#2#3#4#5{
	\fontspec{Technika}
	\bfseries
	\setbox\ctu@title@box\vbox{\parskip0pt\hsize\dimexpr\linewidth-\ctu@title@rulewidth-\ctu@title@rulesep\relax
		\vskip#1
		\raggedright
		#4
		\vskip#2
	}
	#5
	\noindent\begin{tabular}{@{}l@{\hspace*{\ctu@title@rulesep}}l@{}}
		\color{ctubluerule}
		\rule
			[\dimexpr\ht\ctu@title@box-#3-#1]
			{\ctu@title@rulewidth}
			{\dimexpr#1+#3}
	&
		\box\ctu@title@box
	\end{tabular}\hspace*{-10pt}
	\par
}
% the part style, very simply organized
\def\ttlh@ctupt#1#2#3#4#5#6#7#8{
	\bool_gset_true:N \g_ctuthesis_title_haspart_bool
	\ctu@title@boxify{#5}{#3}{
		{#1\strut\ifttl@label#2\fi\par}
		\vskip30pt
		{\color{ctubluetext}#4\strut#8\strut\par}
	}{\dp\ctu@title@box\dimexpr\textheight-\ht\ctu@title@box-\baselineskip\relax
	}
}
\def\ttlh@ctuch#1#2#3#4#5#6#7#8{
	% If the chapter is starred, we still want to \chaptermark it, just we use the variant \chapterstarmark.
	\ifttl@label\else\chapterstarmark{#8}\fi
	% If the chapter is starred, we still want it in the ToC.
	\ifttl@label\else\addcontentsline{toc}{chapter}{#8}\fi
	\ctu@title@boxify{#5}{#3}{
		{#1\strut\ifttl@label#2\fi\par}
		\medskip
		{\color{ctubluetext}#4\strut#8\strut\par}
	}{}
}
% The section style, very simply organized, subsection style differs only by a missing check for star variant and mark inclusion
% Comments to the chapter style apply here too.
\def\ttlh@ctus#1#2#3#4#5#6#7#8{
	\ifttl@label\else\sectionstarmark{#8}\fi	
	\ifttl@label\else\addcontentsline{toc}{section}{#8}\fi
	\ctu@title@boxify@x{#5}{0pt}{#3}{ #1{\ifttl@label#2\fi\strut\color{ctubluetext}#8\strut} }{}
}
\def\ttlh@ctuss#1#2#3#4#5#6#7#8{
	\ifttl@label\else\addcontentsline{toc}{subsection}{#8}\fi
	\ctu@title@boxify@x{#5}{0pt}{#3}{ #1{\ifttl@label#2\fi\strut\color{ctubluetext}#8\strut} }{}
}
\def\ttlh@ctusss#1#2#3#4#5#6#7#8{
	\ifttl@label\else\addcontentsline{toc}{subsubsection}{#8}\fi
	\ctu@title@boxify@x{#5}{0pt}{#3}{ #1{\ifttl@label#2\fi\strut\color{ctubluetext}#8\strut} }{}
}
% Modified \ttl@page@ii to use \cleardoublepage instead of some idiocy
\def\ttl@page@ii#1#2#3#4#5#6#7{%
  \ttl@assign\@tempskipa#3\relax\beforetitleunit
  \if@openright
    \cleardoublepage
  \else
    \clearpage
  \fi
  \@ifundefined{ttl@ps@#6}%
    {\thispagestyle{plain}}%
    {\thispagestyle{\@nameuse{ttl@ps@#6}}}%
  \if@twocolumn
    \onecolumn
    \@tempswatrue
  \else
    \@tempswafalse
  \fi
  \vspace*{\@tempskipa}%
  \@afterindenttrue
  \ifcase#5 \@afterindentfalse\fi
  \ttl@assign\@tempskipb#4\relax\aftertitleunit
  \ttl@select{#6}{#1}{#2}{#7}%
  \ttl@finmarks
  \@ifundefined{ttlp@#6}{}{\ttlp@write{#6}}%
  \vspace{\@tempskipb}%
  \cleardoublepage
  \if@tempswa
    \twocolumn
  \fi
  \ignorespaces}
% Part title setting
\bool_new:N \g_ctuthesis_title_haspart_bool
\titleformat \part [ctupt] {\Huge} {{\huge\partname\nobreakspace}\thepart} {0pt} {\huge} [0.6\textheight]
\titlespacing* \part {0pt} {0pt} {0pt}
% Tweak into titlesec to make part number contain \numberline. They did not include it for some reason.
\def\ttl@tocpart{\def\ttl@a{\protect\numberline{\thepart}}}
% Chapter title setting
\titleformat \chapter [ctuch] {\huge} {{\LARGE\chaptertitlename\ }\thechapter} {-0\dp\strutbox} {\LARGE} [40pt]
\titlespacing* \chapter {0pt} {40pt} {15pt}
% Section title setting
\titleformat \section [ctus] {\Large} {\thesection\quad} {14pt} {} [2pt]
\titleformat \subsection [ctuss] {\large} {\thesubsection\quad} {11pt} {} [0pt]
\titleformat \subsubsection [ctusss] {\normalsize} {\thesubsubsection\quad} {10pt} {} [0pt]
% Paragraph title setting: runin, with a dot.
\cs_new:Nn \ctuthesis_title_adddot:n {#1\@addpunct.\ignorespaces}
\titleformat \paragraph [runin] {\normalfont\normalsize\bfseries\sffamily} {\theparagraph} {1em} {\ctuthesis_title_adddot:n}
\titleformat \subparagraph [runin] {\normalfont\normalsize\itshape} {\thesubparagraph} {1em} {\ctuthesis_title_adddot:n}
\titlespacing* \paragraph {0pt} {2.25ex plus 1ex minus .2ex} {1em plus 0.3em minus 0.2em}
\titlespacing* \subparagraph {\parindent} {1.75ex plus 0.5ex minus .2ex} {0.7em plus 0.3em minus 0.15em}

% Subsections are the lowest numbered titles, and also the lowest included in ToC.
\setcounter{secnumdepth}{2}
\setcounter{tocdepth}{2}

% Appendix treatment: Even if a chapter in appendix has a star, it will get a number (letter in this case), since
% once you get any appendix, it doesn't make sense not to number some of them.
% Users can use \chapter** if-need-be to get an unnumbered chapter in appendix, but we think it's ridiculous.
\cs_set_eq:NN \ctuthesis_title_orig_appendix: \appendix
\cs_set_eq:NN \ctuthesis_title_orig_chapter: \chapter
\cs_set_eq:NN \ctuthesis_title_orig_makeschapterhead: \@makeschapterhead
% an \if that is true only after \appendix
\newif\ifctu@app \ctu@appfalse
% The new \appendix
\DeclareDocumentCommand \appendix { s s } {
	\ctu@apptrue
	% Call the old \appendix
	\ctuthesis_title_orig_appendix:
	% If the document has \part division, let's make "Appendices" a part on their own
	\bool_if:NT \g_ctuthesis_title_haspart_bool { 
		% But only make the notice in ToC if * is used
		\IfBooleanTF { #2 } {
			% Two stars: If nothing in ToC, at least put there a space.
			\addtocontents{toc}{\vskip 2ex plus 0.2pt}
		} {
			\IfBooleanTF { #1 } {
				% One star: Only add "Appendices" in ToC
				\cleardoublepage
				\addcontentsline { toc } { part } { \appendicesname }
			} {
				% Zero stars: Make "Appendices" page (will be in ToC implicitly
				\part*{ \appendicesname }
			}
		}
	}
	% \chapter ignores one star, so that things like {thebibliography} are numbered
	\DeclareDocumentCommand \chapter { s } { \ctuthesis_title_orig_chapter: }
	% another \appendix will generate a warning and be silently ignored
	\DeclareDocumentCommand \appendix { s s } {
		\msg_warning:nn { ctuthesis } { appendix-twice }
	}
}

\msg_new:nnn { ctuthesis } { appendix-twice } { The~macro~ \token_to_str:N \appendix\ should~
	be~used~only~once.~Silently~ignoring. }







%%% TABLE OF CONTENTS, LISTS OF FIGURES AND TABLES

% Right margin in TOC, note that it need not be wider than the page numbers, so we're fine with 1em.
\def\@tocrmarg{1em}

% New dottedtocline, differs from the default by having numberline variable-width and accomodating to the number
\def\ctu@dottedtocline#1#2#3#4#5{\ifnum #1>\c@tocdepth \else
	\addvspace{0pt plus .2pt}
	{
	\leftskip #2\relax
	\rightskip\dimexpr\@tocrmarg plus 1fil
	\parfillskip -\rightskip
	\parindent #3\relax
	\@afterindenttrue
	\interlinepenalty\@M
	\leavevmode
	\null
	\nobreak
	\hskip -\leftskip
	\def\numberline##1{##1\ }
	{#4}
	\nobreak
	\leaders \hbox {$\m@th \mkern 2mu\hbox {.}\mkern 2mu$}\hfill
	\nobreak
	{\kern0.5em#5}
	\par
	}
\fi}

% Part in ToC: centered, with no pagenumber, bf sf, and with nice space above.
\def\l@part#1#2{
	\addvspace{2ex plus .2pt}
	{
	\centering
	\@afterindenttrue
	\interlinepenalty\@M
	\leavevmode
	\null
	\nobreak
	\def\numberline##1{\partname \nobreakspace ##1\\}%
	\sffamily \bfseries
	#1
	\par
	}\penalty10000\relax
}
% Chapter in ToC: bf sf, no leaders, completely left-aligned, the numberline is variable-width,
% nice space above the chapter line
\def\l@chapter#1#2{
	\addvspace{1ex plus .2pt}
	{
	\raggedright
	\rightskip\dimexpr\@tocrmarg plus 1fil
	\parfillskip -\rightskip
	\@afterindenttrue
	\interlinepenalty\@M
	\leavevmode
	\null
	\nobreak
	\def\numberline##1{##1\hspace*{0.5em}}
	\sffamily \bfseries
	#1
	\nobreak
	\hfill
	\nobreak
	{\kern0.5em#2}
	\par
	}\penalty9999\relax
}
% The section and lower in Toc. We hope nobody would put `\paragraph`s in ToC
\def\l@section{\ctu@dottedtocline{1}{0.6em}{0em}}
\def\l@subsection{\ctu@dottedtocline{2}{1.2em}{0.6em}}
\def\l@subsubsection{\ctu@dottedtocline{3}{1.8em}{1.2em}}
% Figures and tables in LoF and LoT exactly as sections, just the level is set to -2 so that they're always shown
\def\l@figure{\ctu@dottedtocline{-2}{0.6em}{0em}}
\let\l@table\l@figure






%%% FANCY HEADERS

\RequirePackage{fancyhdr}

% Spacial pagestyle for twocolumnpages in the frontmatter: includes centered pagenumber in the bottom
% and we abuse \fancyhead[C] to include the dividing rule
\fancypagestyle{twocol}{
	\fancyhf { }
	\fancyfoot [ RO, LE ] { \texttt { \ctufield { preprint } } }
	\fancyfoot [ C ] { \thepage }
	\fancyhead [ C ] {
		\leavevmode
		\smash {
			\color{ctubluerule}
			\rule [ \dimexpr - \textheight - \headsep - 6pt ]{ 11pt }{ \textheight } 
		}
	}
	\renewcommand { \headrulewidth } { 0pt }
	\renewcommand { \footrulewidth } { 0pt }
	\dim_set_eq:NN \headwidth \textwidth
}

\fancypagestyle { cleardoublepage } {
	% On \cleardoublepage (without star), we want the page number, but nothing else,
	% that is, the same as plain pagestyle
	\ps@plain
}

% Plain pagestyle for chapter and part titles: no rules, just the page number
\fancypagestyle { plain } {
	\fancyhf { }
	\fancyfoot [ RO, LE ] { \texttt { \ctufield { preprint } } }
	\fancyfoot [ C ] { \thepage }
	\renewcommand \headrulewidth { 0pt }
	\renewcommand \footrulewidth { 0pt }
	% reset headwidth to the global textwidth
	\dim_set_eq:NN \headwidth \g_ctuthesis_geometry_textwidth_dim
}

% Headings pagestyle for standard text pages
\fancypagestyle { headings } {
	% Copy the plain pagestyle
	\ps@plain
	% Include the left/rightmark and leaders
	\fancyhead [ LE ] { \ctuthesis_fancy_xlap:nn { right } { \leftmark \ctuthesis_fancy_leaders:n { l } } }
	\fancyhead [ RO ] { \ctuthesis_fancy_xlap:nn { left } { \ctuthesis_fancy_leaders:n { r } \rightmark } }
}

% The leaders macro for the fancyhead: nice little boxes. We make three of them: left, right and center-aligned

\box_new:N \g_ctuthesis_fancy_bluerule_l_box
\hbox_set_to_wd:Nnn \g_ctuthesis_fancy_bluerule_l_box { 10.5pt } {
	{ \color{ctubluedot} \smash{\fontsize{28.88}{0}\fontfamily{lmss}\bfseries.} }
	\hfil
}

\box_new:N \g_ctuthesis_fancy_bluerule_r_box
\hbox_set_to_wd:Nnn \g_ctuthesis_fancy_bluerule_r_box { 10.5pt } {
	\hfil
	{ \color{ctubluedot} \smash{\fontsize{28.88}{0}\fontfamily{lmss}\bfseries.} }
}

\box_new:N \g_ctuthesis_fancy_bluerule_c_box
\hbox_set_to_wd:Nnn \g_ctuthesis_fancy_bluerule_c_box { 10.5pt } {
	\hfil
	{ \color{ctubluedot} \smash{\fontsize{28.88}{0}\fontfamily{lmss}\bfseries.} }
	\hfil
}

% The leaders macro for the fancyhead: nice little boxes,
% the parameter #1 chooses one of the boxes above
\cs_new_protected:Nn \ctuthesis_fancy_leaders:n {
	\hspace*{-1pt}
	\leaders \box_use:c {g_ctuthesis_fancy_bluerule_#1_box} \hfill
	\hspace*{-1pt}
}

% This macro works as follows: The width of the material is actually zero. We use \hbox_overlap_left/right
% to make the headers as wide as we wish, the direction is in #1. In #2 is the contents, and we suppose
% that it contains \ctuthesis_fancy_leaders: and \left/rightmark in desired order; it's typeset gray sf it
\cs_new_protected:Nn \ctuthesis_fancy_xlap:nn {
	\color{ctugray} \sffamily\itshape
	\leavevmode \use:c { hbox_overlap_#1:n } {
		\hbox_to_wd:nn { \paperwidth - 1in - \evensidemargin + 1em } { #2 }
	}
}

% We rename the original \mark... commands and disable them; we do it ourselves
\cs_set_eq:NN \ctuthesis_fancy_markboth_orig:nn \markboth
\cs_set_eq:NN \ctuthesis_fancy_markright_orig:n \markright
\let\markboth\@gobbletwo
\let\markright\@gobble

% New chapter and section marks with no \MakeUppercase
\renewcommand \chaptermark [1] { \ctuthesis_fancy_markboth_orig:nn { \thechapter.\ #1 }{ \thechapter.\ #1 } }
\renewcommand \sectionmark [1] { \ctuthesis_fancy_markright_orig:n { \thesection.\ #1 } }

% Newly defined \chapter/sectionstarmark to be used with unnumbered chapters and sections.
% Not that we like unnumbered ones, but if they need to be there, they sould be in headings, too.
\newcommand \chapterstarmark [1] { \ctuthesis_fancy_markboth_orig:nn { #1 } { #1 } }
\newcommand \sectionstarmark [1] { \ctuthesis_fancy_markright_orig:n { #1 } }

% Better \cleardoublepage that takes proper treatment of the empty pages.
% The starred variant leaves the empty page inbetween really completely empty
\DeclareDocumentCommand \cleardoublepage { s } {
	\clearpage
	\if@twoside\ifodd\c@page\else
%		\IfBooleanTF { #1 } {
			% Starred variant: completely empty page
			\leavevmode
			\thispagestyle{empty}
%		} {
			% Non-starred variant: include page number
			\leavevmode
			\thispagestyle{cleardoublepage}
%		}
		\newpage
		% In twocolumn, we need \newpage twice
		\if@twocolumn\hbox{}\newpage\fi
	\fi\fi
}






%%% FLOATS

\usepackage{float,caption}
% default placement of all floats is `t`
\floatplacement{figure}{t}
\floatplacement{table}{t}
\floatplacement{figure*}{t}
\floatplacement{table*}{t}
% minimum 70% of page height on float pages
\renewcommand{\floatpagefraction}{0.7}
% 2 floats on top, no on bottom
\setcounter{topnumber}{2}
\setcounter{bottomnumber}{0}
\setcounter{totalnumber}{4}
% up to 90% of floats on a mixed page, at least 7% of text
\renewcommand{\topfraction}{0.7}
\renewcommand{\bottomfraction}{0.7}
\renewcommand{\textfraction}{0.2}
% float caption: justification with extra margins on both sides
\DeclareCaptionJustification{myjustify}{\leftskip.8em \rightskip.8em \parfillskip.8em plus 1fil}
% float caption: special format
\DeclareCaptionFormat{myformat}{%
  % float name bf sf
  \begingroup\sffamily\bfseries#1#2:~\endgroup
  % float caption normal font
  \begingroup#3\par\endgroup
}
\DeclareCaptionFormat{mysubformat}{%
  % float name small caps
  #1#2 %
  % float caption normal font
  \begingroup#3\par\endgroup
}
% plain float style is what we like, with captions at the bottom
\floatstyle{plain}
% restyle already existing floats
\restylefloat{figure}
\restylefloat{table}
% set up the captions: no indent of captions, no seperator (seperator is handled by {myformat})
\captionsetup{format=myformat,indent=0pt,labelsep=none,justification=myjustify,font={small},position=below}
% float-caption distance
\abovecaptionskip 8pt plus 3pt minus 1pt
% graphics allowed: pdf, png, jpg
\DeclareGraphicsExtensions{.pdf,.png,.jpg}
% floats should be centered by default:
\g@addto@macro{\@floatboxreset}{\centering}
% nice table rules
\usepackage{booktabs}
\newcommand \Midrule { \midrule[\heavyrulewidth] }
% colored tables
\newenvironment{ctucolortab}{\begin{lrbox}{0}}{\end{lrbox}\ht0\dimexpr\ht0 + 2pt\relax{\fboxsep0pt\colorbox{ctubluebg}{\usebox{0}}}}
\newenvironment{ctucolortab*}{\begin{lrbox}{0}}{\end{lrbox}{\fboxsep0pt\colorbox{ctubluebg}{\usebox{0}}}}






%%% LISTS

% Temp box
\box_new:N \g_ctuthesis_tempa_box
% Macro used to generate the itemize labels
% parameters 1:depth, 2: raise (without ex), 3: scale, 4:color
\cs_new_protected:Nn \ctuthesis_list_prepare_labelitem:nnnn {
	% Each label stored in a box
	\box_new:c { g_ctuthesis_list_label_#1_box }
	% use the temp box to measure the symbol
	\hbox_gset:Nn \g_ctuthesis_tempa_box {
		\scalebox{#3}{\fontfamily{lmss}\bfseries\color{#4}.}
	}
	% The label is actually a shifted up, scaled, coloured lmss bf dot
	\hbox_gset:cn { g_ctuthesis_list_label_#1_box } {
		\box_move_up:nn { #2 ex } \hbox_to_wd:nn {
			\box_ht:N \g_ctuthesis_tempa_box
		} {
			\hss
			\scalebox{#3}{\fontfamily{lmss}\bfseries\color{#4}.}
			\hss
		}
	}
	\cs_set:cpn { labelitem #1 } { \box_use:c { g_ctuthesis_list_label_#1_box } }
}
% The four labels for itemize
\ctuthesis_list_prepare_labelitem:nnnn { i } { 0.18 } { 3.6 } { ctublueitemi }
\ctuthesis_list_prepare_labelitem:nnnn { ii } { 0.18 } { 3.6 } { ctublueitemii }
\ctuthesis_list_prepare_labelitem:nnnn { iii } { 0.36 } { 2.4 } { black }
\ctuthesis_list_prepare_labelitem:nnnn { iv } { 0.48 } { 1.2 } { black }

% Macro for making the box with lmss bf dot "." scaled to the size of \strutbox
% #1 is the width of the box (relative to the width of the resulting dot) and
% #2 is the color of the box
\cs_new_protected:Nn \ctuthesis_list_enumdot:nn {
	\hbox_to_wd:nn { #1\ht\strutbox + #1\dp\strutbox } {
		\hss
		\raisebox{-\dp\strutbox}{
			\resizebox{!}{\dimexpr\ht\strutbox+\dp\strutbox}{
				\fontfamily{lmss}\bfseries\color{#2}.
			}
		}
		\hss		
	}
}

% \itembox takes 5 parameters.
% (#1)(#2) are the left and right padding
% [#3][#4] are the box color and text color
% #5 is the text
\DeclareDocumentCommand \itembox { D(){0.15em} D(){0.15em} >{\TrimSpaces}O{ctubluebg} >{\TrimSpaces}O{black} >{\TrimSpaces}m } {
	\ooalign{
		\hfil \ctuthesis_list_enumdot:nn {1} {#3}
	\cr
		\ctuthesis_list_enumdot:nn {1} {#3} \hfil
	\cr
		\skip_horizontal:n { 0.34\ht\strutbox + 0.34\dp\strutbox }
		\cleaders \ctuthesis_list_enumdot:nn {0.34} {#3} \hfil
		\skip_horizontal:n { 0.34\ht\strutbox + 0.34\dp\strutbox }
	\cr
		\hfil
		\hbox:n { \hspace{#1} { \color{#4} #5 } \hspace{#2} }
		\hfil
	}
}

% Redefine the enumerate labels. The first two levels are boxed,
% the other two are not.
%\def\labelenumi{{\normalfont\itembox[ctubluedarkbg][white]{\bfseries\theenumi.}}}
\def\labelenumi{{\normalfont\itembox{\theenumi.}}}
\def\labelenumii{{\normalfont\itembox(0.1em)(0.1em){\theenumii.}}}
\def\labelenumiii{{\normalfont(\theenumiii)}}
\def\labelenumiv{{\normalfont(\theenumiv)}}

\leftmargini22.2pt
\labelsep6pt







%%% MATH, THEOREMS

\usepackage{amsmath,amssymb,amsfonts}

% Optional loading of amsthm
\ctuthesis_if_switch:nT { pkg-amsthm } {

	\usepackage{amsthm}

	% Redefine standard styles: plain is slanted, definition is normal, note has an italic head (like proof)
	\newtheoremstyle{plain}{3pt plus 1pt}{3pt plus 1pt}{\slshape}{}{\bfseries\sffamily}{.}{.5em}{}
	\newtheoremstyle{plainit}{3pt plus 1pt}{3pt plus 1pt}{\slshape}{}{\bfseries\sffamily}{.}{.5em}{}
	\newtheoremstyle{definition}{3pt plus 1pt}{3pt plus 1pt}{}{}{\bfseries\sffamily}{.}{.5em}{}
	\newtheoremstyle{note}{3pt plus 1pt}{3pt plus 1pt}{}{}{\itshape}{.}{.5em}{}

	% Default style: plain
	\theoremstyle{plain}
	
	% We redefine proof to include a more reasonable spacing above it.
	\DeclareDocumentEnvironment { proof } { O{\proofname} } {
		\par
		\pushQED{\qed}
		\normalfont \topsep0pt plus 1pt
		\trivlist
		\item[
			\hskip\labelsep
		    \itshape
			#1\@addpunct{.}
		]
		\ignorespaces
	}{
		\popQED\endtrivlist\@endpefalse
	}

}	






%%% INDEX

\ctuthesis_if_switch:nT { pkg-makeidx } {

	% Initialize makeindex
	\usepackage{makeidx}
	\makeindex

	% We use multicol rather than twocolumn for the index
	\usepackage{multicol}
	
	% Redefine the index environment to just make the header and start the multicols* envrionment
	% (starred because we don't want to balance the columns), then redefine \item as in original theindex
	\DeclareDocumentEnvironment { theindex } { } {
		\chapter* {\indexname }
		\begin{multicols*}{2}
		\let\item\@idxitem
		% Do not put index sections in ToC. We save the tocdepth and set it to zero, only to restore it after.
		\addtocontents{toc}{\protect\ctu@savetocdepth}
		\addtocontents{toc}{\protect\setcounter{tocdepth}{0}}
		% Inside index, the sections are tighter together
		\titlespacing*\section{}{10pt}{4pt}
	} {
		\addtocontents{toc}{\protect\ctu@restoretocdepth}
		\end{multicols*}
		\clearpage
	}
	
	% How does this work: in \ctu@tocdepth we save the current tocdepth
	\newcommand\ctu@savetocdepth{\xdef\ctu@saved@tocdepth{\the\c@tocdepth}}
	% This macro restores the saved value.
	\newcommand\ctu@restoretocdepth{\global\c@tocdepth\ctu@saved@tocdepth\relax}
}





%%% LISTINGS

\ctuthesis_if_switch:nT { pkg-listings } {
	\RequirePackage{listings}
	\lstset{
		backgroundcolor = \color{ctubluebg},
		basicstyle = \ttfamily,
		columns = fullflexible,
		keepspaces = true,
		tabsize = 4,
%		frame = leftline,
%		framesep = 0pt,
%		framexleftmargin = 3pt,
%		framerule = 0.667pt,
%		rulecolor = \color{ctubluerule},
		frame = single,
		framesep = 0pt,
		framexleftmargin = 2pt,
		framerule = 2pt,
		rulecolor = \color{ctubluebg},
		framexrightmargin = 2pt,
		xleftmargin = 4pt,
		xrightmargin = 4pt,
	}
	\cs_new_eq:NN \ctuthesis_lst_orig_lstinline:w \lstinline
	\newsavebox \ctuthesis_lst_box
	\newcommand* \ctulstsep {0.2em}
	\DeclareDocumentCommand \ctuthesis_lst_do_ctulstinline:nnw { m m m v } {
		\leavevmode
		\microtypesetup{activate=false}
		\begin{lrbox}{\ctuthesis_lst_box}
		\ctuthesis_lst_orig_lstinline:w[#2]#3#4#3
		\end{lrbox}
		\str_if_eq:xnTF {#1} {none}
			{{ \hbox:n {\strut\hspace*{\ctulstsep}\box\ctuthesis_lst_box\hspace*{\ctulstsep}} }}
			{{ \fboxsep0pt\colorbox{#1} {\strut\hspace*{\ctulstsep}\box\ctuthesis_lst_box\hspace*{\ctulstsep}} }}
		\microtypesetup{activate=true}
	}
	\DeclareDocumentCommand \ctulst { D(){\g_ctuthesis_ctulstbg_tl} O{} m } {
		\ctuthesis_lst_do_ctulstinline:nnw { #1 } { #2 } { #3 } #3
	}
	\DeclareDocumentCommand \ctulstbr { D(){ctulstbg} O{} m } {
		\leavevmode
		\begin{lrbox}{\ctuthesis_lst_box}
		\ctuthesis_lst_orig_lstinline:w[#2]{#3}
		\end{lrbox}
		\colorbox{#1}{\box\ctuthesis_lst_box}
	}
}






%%% HYPERREF

\ctuifswitch { pkg-hyperref } {
	\usepackage[
		pdfpagelayout=TwoPageRight, % Correct twopage layout with recto on the right
	]{hyperref}
	% Links with orange borders, in ToC only the page is linked
	\hypersetup{
		colorlinks=false,
		linktocpage=true,
		allbordercolors=ctuorange,
	}
	% Prepare PDFinfo
	\tl_new:N \g_ctuthesis_hyperref_pdfinfo_tl
	\tl_set:Nx \g_ctuthesis_hyperref_pdfinfo_tl {
		pdfinfo = {
			Title = \ctuthesis_field_use:nn { title } { title },
			Author = \ctuthesis_field_use:n { author },
		}
	} 
	\exp_args:NV \hypersetup \g_ctuthesis_hyperref_pdfinfo_tl
}{}

%\ctuifswitch { pkg-hyperref } {
%	\hypersetup{colorlinks=false,linktocpage=true,linkbordercolor=0 122 195}
%}






%%% VARIOUS TWEAKS

% Provide \phantomsection and \texorpdfstring to make them available in case hyperref wasn't loaded
\providecommand\phantomsection{}
\providecommand\texorpdfstring[2]{#1}
% Provide \microtypesetup to make it available in case microtype wasn't loaded
\providecommand\microtypesetup[1]{}








\endinput
